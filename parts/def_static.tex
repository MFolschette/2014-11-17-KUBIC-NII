% Présentation de l'analyse statique

\begin{frame}
  \frametitle{Static analysis: successive reachability}
  \framesubtitle{\tcite{\cpmrmscs}}

Successive reachability of processes:

\begin{columns}
\begin{column}{0.55\textwidth}

\begin{center}
\scalebox{0.75}{
\begin{tikzpicture}
  %\path[use as bounding box] (-1,-3) rectangle (7,2);
  \exatt

  \TState{1-5}{a_0,b_0,c_0,d_0}
  \TState{6}{a_0,b_0,c_1,d_0}
  \TState{7}{a_0,b_0,c_1,d_1}
  \TState{8}{a_0,b_1,c_1,d_1}
  \node<9>[process,very thick] (d_2) at (d_2.center) {};
  \TState{9}{a_0,b_1,c_1,d_2}

  \node<2>[objective] at (d_1.center) {1?};
  \node<2>[objective] at (d_2.center) {2?};

  \node<3>[objective] at (d_1.center) {1?};
  \node<3>[objective] at (b_1.center) {2?};
  \node<3>[objective] at (d_2.center) {3?};

  \node<4-8>[objective] at (d_2.center) {1?};

  %\node<3>[process,very thick] (d_1) at (d_1.center) {1?};
  %\node<3>[process,very thick] (b_1) at (b_1.center) {2?};
  %\node<3>[process,very thick] (d_2) at (d_2.center) {3?};

  %\node<4-8>[process,very thick] (d_2) at (d_2.center) {1?};

  \only<5>{\THit{a_0}{hlhit}{c_0}{.north}{c_1}}
  \path<5>[bounce,bend left,hlhit] \TBounce{c_0}{}{c_1}{.west};
  \only<6>{\THit{b_0}{hlhit}{d_0}{.west}{d_1}}
  \path<6>[bounce,bend left,hlhit] \TBounce{d_0}{}{d_1}{.south};
  \only<7>{\THit{c_1}{bend left=20pt,hlhit}{b_0}{.west}{b_1}}
  \path<7>[bounce,bend left,hlhit] \TBounce{b_0}{}{b_1}{.south};
  \only<8>{\THit{b_1}{hlhit}{d_1}{.west}{d_2}}
  \path<8>[bounce,bend left,hlhit] \TBounce{d_1}{}{d_2}{.south};
\end{tikzpicture}
}
\end{center}

\end{column}
\begin{column}{0.45\textwidth}

%\pause
~\\~\\
\begin{itemize}
  \item Initial state
    \\ \rex{\PHetat{a_1, b_0, c_0, d_0}} \pause
  \item Objectives
    \\ \rex{$[\ \Rsh d_1\ \PHconcat\ \Rsh d_2\ ]$} \pause
    \\\smallskip \rex{$[\ \Rsh d_1 \PHconcat\ \Rsh b_1 \PHconcat\ \Rsh d_2\ ]$} \pause
    \\\smallskip \rex{$[\ \Rsh d_2\ ]$} \pause
\end{itemize}

\end{column}
\end{columns}

\medskip
\begin{center}
\f Concretization of the objective = scenario

\ex{
$ \only<5>{\underline{\PHfrappe{a_0}{c_0}{c_1}}} \only<-4,6->{\PHfrappe{a_0}{c_0}{c_1}} \PHconcat %
  \only<6>{\underline{\PHfrappe{b_0}{d_0}{d_1}}}\only<-5,7->{\PHfrappe{b_0}{d_0}{d_1}} \PHconcat %
  \only<7>{\underline{\PHfrappe{c_1}{b_0}{b_1}}}\only<-6,8->{\PHfrappe{c_1}{b_0}{b_1}} \PHconcat %
  \only<8>{\underline{\PHfrappe{b_1}{d_1}{d_2}}}\only<-7,9->{\PHfrappe{b_1}{d_1}{d_2}}
$}
\end{center}
\end{frame}



\begin{frame}
  \frametitle{Approximations for the Reachability Analysis}
  \framesubtitle{\tcite{\cpmrmscs}}

Check reachability properties:
\begin{center}
  «~From an initial state $s_0$, is it possible to reach a state $s_n$ where $a_i$ is active?~»
\end{center}
Approximations: $P$ and $Q$, built so that \tval{$P \Rightarrow R \Rightarrow Q$}

\begin{center}
\scalebox{0.6}{
\begin{tikzpicture}
  \path[use as bounding box] (-5,-3.5) rectangle (5,3.5);
  \definecolor{r2}{RGB}{238,10,38}

  %\path<2->[shading=1, inner color=r2, outer color=white] (3.5,-2.8) -- (4.4,3.2) -- (0,3) -- (-4.5,1.4) -- (-2.5,-2.5) -- (0,-3.6) -- (2.8,-2.8);
  \draw<2->[shading=2, inner color=r2, outer color=white, rounded corners, draw=none] (-6,3.5) rectangle (6,-3.5);
  %\draw<2->[thick,fill=white] (2.5,-2.1) -- (3,2.5) -- (-2.7,1.3) -- (-2,-2) -- (2.5,-2.1);
  \draw<2->[thick,fill=white] (-2.8,2) rectangle (2.8,-2);
  %\draw<6->[thick,fill=lightyellow] (2.5,-2.1) -- (3,2.5) -- (-2.7,1.3) -- (-2,-2) -- (2.5,-2.1);
  \draw<6->[thick,fill=lightyellow] (-2.8,2) rectangle (2.8,-2);

  \node<2->[text width=3.5cm, color=red] (s1) at (-5,2) {Over-approximation};
  \path<2->[->,very thick,color=red] (s1.south) edge (-3,1.2);
  \node<2->[text width=3cm,color=black] (q) at (2.2,2.3) {$\neg Q$};

  %\draw<4->[thick, shading=1, top color=darkgreen, bottom color=green] (.5,-.8) -- (1,0) -- (.3,1) -- (-1,.5) -- (-.5,-.5) -- (.5,-.8);
  \draw<4->[thick, shading=1, top color=darkgreen, bottom color=green] (-1.5,.7) rectangle (1.5,-.7);;
  \node<4->[text width=3.5cm,color=darkgreen] (s2) at (5.2,-2.5) {Under-approximation};
  \node<4->[text width=3cm,color=black] (p) at (1.8,.4) {$P$};
  \path<4->[->,very thick,color=darkgreen] (s2) edge (1,-.8);

  \node[text width=3cm,align=center,color=darkcyan] (s) at (0,-1.7) {Exact solution};
  \node<1->[text width=3cm,color=darkcyan] (s0) at (0,0) {};
  \draw[color=darkcyan, ultra thick] (0,0) ellipse (2 and 1.5);
  \node[text width=3cm,color=black] (r) at (2,1.2) {$R$};

  \only<3->{
    \node[point] at (-3.3,-1) {};
    \node[point] at (2,2.5) {};
  }
  \only<5->{
    \node[point] at (-1,.2) {};
    \node[point] at (1.1,-.5) {};
  }
  \only<7->{
    \node[point] at (-.5,-1.1) {};
    \node[point] at (2.5,1) {};
  }
\end{tikzpicture}
}
\end{center}

\uncover<8->{
Polynomial complexity in the number of sorts\\
Exponential complexity in the number of processes in each sort
\begin{fleches}
  \item Efficient for big models with few expression levels
\end{fleches}
}
\end{frame}



\begin{frame}[c]
  \frametitle{Implementation \& Execution times}

\Pint\tval{: Existing free OCaml library}

\medskip
\f Compiler + tools for Process Hitting models

\f Documentation \& examples: \lien{https://github.com/pauleve/pint}

\pause
\bigskip
\medskip
\tval{Computation time for various reachability analyses:}

\medskip
\small
\begin{tabular}{r||c|c|c|c||c|c|c|}
\hline
\tval{Model} & Sorts & Procs & Actions & States & Biocham$^1$ & libddd$^2$ & \cellcolor{couleurtitre} \Pint \\\hline
\tval{\ex{egfr20}} & 35 & 196 & 670 & $2^{64}$ & [3s--$\infty$] & [1s--150s] & \cellcolor{couleurtitre} \tval{0.007s} \\\hline
\tval{\ex{tcrsig40}} & 54 & 156 & 301 & $2^{73}$ & [1s--$\infty$] & [0.6s--$\infty$] & \cellcolor{couleurtitre} \tval{0.004s} \\\hline
\tval{\ex{tcrsig94}} & 133 & 448 & 1124 & $2^{194}$ & $\infty$ & $\infty$ & \cellcolor{couleurtitre} \tval{0.030s} \\\hline
\tval{\ex{egfr104}} & 193 & 748 &  2356 & $2^{320}$ &  $\infty$ & $\infty$ & \cellcolor{couleurtitre} \tval{0.050s}\\\hline
\end{tabular}

\medskip
\quad$^1$ Inria Paris-Rocquencourt/Contraintes\\
\quad$^2$ LIP6/Move

\cmodels
\end{frame}
