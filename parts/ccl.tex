% Conclusion

\begin{frame}[c]
  \frametitle{General Conclusion}

Standard Process Hitting allows to represent biological regulatory networks
in an \tval{atomistic} fashion:
\begin{itemize}
  \item Existing efficient static analysis
  \item But temporal shift issues
  \item Limited modeling power
\end{itemize}

\medskip
\tval{Extensions of the Process Hitting} to improve the expressivity:
\begin{itemize}
  \item Rectification of the temporal shift \f Strictly higher expressivity
  \item Allows to abstract temporal parameters
  \item New links to other formalisms (Thomas, PN, etc.)
\end{itemize}

\medskip

\tval{Static analysis} of the Canonical Process Hitting:
\begin{itemize}
  \item Efficient analysis of reachability properties
  \item Applicable to the extensions at the cost of a translation
  \item New kind of property: simultaneous activation
\end{itemize}

% 
% 
% Process Hitting: an atomistic modeling with powerful static analysis
% 
% \medskip
% \begin{enumerate}[1.]
%   \item Stochastic parameters:
%     \begin{itemize}
%       \item To model systems with chronometric features
%       \item \tval{Continuous time}
%       \item But \tval{hard to analyze}
%     \end{itemize}
%   \item Classes of priorities:
%     \begin{itemize}
%       \item Allows to reproduce the same behaviors
%       \item Efficient \tval{static analysis}
%       \item But the translation to canonical form faces \tval{combinatorial explosion}
%     \end{itemize}
%   \item Neutralizing edges:
%     \begin{itemize}
%       \item Alternative to priorities
%       \item Closer to reality in some cases
%       \item \tval{Lighter translation} to canonical form
%     \end{itemize}
% \end{enumerate}
% 
% \vfill
% \Large
% \begin{flushright}
%   \tval{Thank you}\hspace{1cm}~
% \end{flushright}
% \vfill

\end{frame}

\setbeamercovered{invisible}



\begin{frame}[c]
  \frametitle{Outlooks}

New \tval{exploitation} possibilities:
\begin{itemize}
  \item Modeling and analysis of full databases
  \item Study of uncontrollable behaviors or punctual perturbations
  \item Research of interesting properties (attractors, oscillations, ...)
\end{itemize}

\medskip
Improvement of the \t{static analysis}:
\begin{itemize}
  \item Refining in order to reduce the non-conclusiveness
  \item New methods using by-products such as the local causality graph
  \item New properties to check (temporal logic, counters, ...)
\end{itemize}

\medskip
Enrichment of the \tval{modeling power}:
\begin{itemize}
  \item Dynamical classes of priorities
  \item Guarded actions or complex logic gates
  \item New model checking tools (Hoare logic, ...)
\end{itemize}

\end{frame}



\begin{frame}[c]
  \frametitle{Collaborations}

Participation to the \tval{ANR blanc} project \tval{BioTempo} (March 2011 -- November 2014):
\begin{center}
“Language, time representations and hybrid models\\
for the analysis of incomplete models in molecular biology”
\end{center}
Task 3: Introduce synchronization and continuous time in chronological models:\\
programming language, multi-clocks and hybrid systems

\bigskip
\bigskip
3 months PhD internship (March -- May 2012):\\
\tval{National Institute of Informatics} (Tokyo, Japan)\\
Invited in the team of \tval{Katsumi Inoue}
\begin{center}
“Automated Reasoning and Hypothesis\\
Finding for Systems Biology”
\end{center}
Partnership organized with AtlanSTIC
Financial participation of Centrale Initiatives 



%\tval{Inoue Laboratory} (NII, Sokendai): Constraint Programming, Systems Biology

%\tval{MeForBio} (IRCCyN, ÉCN): Formal Methods for Bioinformatics

%\tval{AMIB} (LIX, Polytechnique): Algorithms and Models for Integrative Biology

% \bigskip\footnotesize
% \begin{center}
%   $\left.\text{\begin{tabular}{ccc}
%       \includegraphics[height=1.5cm]{figs/Olivier.jpg}
%     & \includegraphics[height=1.5cm]{figs/Morgan.jpg} \\
%       \tval{Olivier ROUX} & \tval{Morgan MAGNIN} \\
%       Professeur \& chef d'équipe & Maître de conférences
%   \end{tabular}}\right\}$ %\text{\tval{MeForBio}}$%}
%   \parbox{2cm}{\tval{MeForBio}\\IRCCyN\\(Nantes, France)}
% 
%   \vspace*{3em}
%   $\left.\text{\begin{tabular}{c}
%     \includegraphics[height=1.5cm]{figs/Loic.jpg} \\ \tval{Loïc PAULEVÉ} \\ Chargé de recherche CNRS
%   \end{tabular}}\right\}$%\text{\tval{AMIB}}$
%   \parbox{1.5cm}{\tval{Bioinfo/AMIB}\\LRI\\(Orsay, France)}
%   \hspace*{3em}
%   $\left.\text{\begin{tabular}{c}
%     \includegraphics[height=1.5cm]{figs/Inoue-sensei.jpg} \\ \tval{Katsumi INOUE} \\ Professeur \& chef d'équipe
%   \end{tabular}}\right\}$%\text{\tval{Inoue Laboratory}}$
%   \parbox{1.5cm}{\tval{Inoue Lab.}\\NII\\(Tokyo, Japon)}
% \end{center}

\end{frame}



\begin{frame}[c]
  \frametitle{Personal Contributions}

\small
\emphcolor{Book chapter:}
\begin{itemize}
  \item Paulevé, Chancellor, \tval{Folschette}, Magnin, Roux :\\
    \tval{Analyzing Large Network Dynamics with Process Hitting},\\
    \textit{Logical Modeling of Biological Systems}, août 2014
\end{itemize}

\medskip
\emphcolor{Conferences and workshops:}
\begin{itemize}
  \item \tval{Folschette}, Paulevé, Magnin, Roux :\\
    \tval{Under-approximation of reachability in multivalued asynchronous networks},
    \begin{tikzpicture}
      \path[use as bounding box] (0,0) rectangle (0,.1);
      \path[grosarca, ->] (2,0.08) edge (.3,0.08);
      \path[grosarca, ->] (2,-3.45) edge (.3,-3.45);
      \path[grosarca] (2,0.08) edge (2,-3.45);
    \end{tikzpicture}\\
    CS2Bio'13, \textit{Electronic Notes in Theoretical Computer Science}, \vol 299, 2013\\
    \emphcolor{selected for a special issue} of \textit{Theoretical Computer Science}
 \item \tval{Folschette}, Paulevé, Inoue, Magnin, Roux :\\
    \tval{Concretizing the process hitting into biological regulatory networks},
    \begin{tikzpicture}
      \path[use as bounding box] (0,0) rectangle (0,.1);
      \path[grosarcb, ->] (2,0.08) edge (0.3,0.08);
      \path[grosarcb, ->] (2,-0.9) edge (0.8,-0.9);
      \path[grosarcb, ->] (2,-3.15) edge (1.1,-3.15);
      \path[grosarcb] (2,0.08) edge (2,-3.15);
    \end{tikzpicture}\\
    CMSB'12, \textit{Lecture Notes in Computer Science}, 2012
  \item \tval{Folschette}, Paulevé, Inoue, Magnin, Roux :\\
    \tval{Abducing Biological Regulatory Networks from Process Hitting models},\\
    \textit{ECML-PKDD'12 / LDSSB'12}, 2012
\end{itemize}

\emphcolor{Current journal submissions:}
\begin{itemize}
  \item \tval{Folschette}, Paulevé, Magnin, Roux :\\
    \tval{Sufficient Conditions for Reachability in Automata Networks with Priorities},\\
    \emphcolor{submitted} to a special issue of \textit{Theoretical Computer Science}
  \item \tval{Folschette}, Paulevé, Inoue, Magnin, Roux :\\
    \tval{Constructing Biological Regulatory Networks from Process Hitting models},\\
    \emphcolor{in revision} for \textit{Theoretical Computer Science}
  \item Paulevé, \tval{Folschette}, Magnin, Roux :\\
    \tval{Analyses statiques de la dynamique des réseaux d'automates indéterministes},\\
    \emphcolor{submitted} to a special issue of \textit{Technique et Science Informatiques}
\end{itemize}

\end{frame}



% 
% \begin{frame}[c]
%   \frametitle{Contributions personnelles}
% 
% \small
% \emphcolor{Chapitre de livre} avec Paulevé, Chancellor, Magnin, Roux :
% \begin{itemize}
%   \item \tval{Analyzing Large Network Dynamics with Process Hitting},\\
%     \textit{Logical Modeling of Biological Systems},
% %    éditeurs : Luis Farinas del Cerro et Katsumi Inoue,
%     2014%, ISBN 978-1-84821-680-8.
% \end{itemize}
% 
% \medskip
% \emphcolor{Workshop} avec Paulevé, Magnin, Roux :
% \begin{itemize}
%   \item \tval{Under-approximation of reachability in multivalued asynchronous networks},\\
%     CS2Bio'13,
%     %in: Proceedings of the fourth International Workshop on Interactions between Computer Science and Biology,
%     %éditeurs : Emanuela Merelli et Angelo Troina,
%     \textit{Electronic Notes in Theoretical Computer Science}, \vol 299, 2013\\
%     \emphcolor{sélectionné pour un numéro spécial} de \textit{Theoretical Computer Science}
%     %33--51, Springer Berlin Heidelberg, juin 2013, DOI 10.1016/j.entcs.2013.11.004.
%     %\emphcolor{Selected for a special issue in the journal \textit{Theoretical Computer Science}.}
% \end{itemize}
% 
% \emphcolor{Conférence et workshop} avec Paulevé, Inoue, Magnin, Roux :
% \begin{itemize}
%   \item \tval{Concretizing the process hitting into biological regulatory networks},\\ %\newline{}
%     CMSB'12, \textit{Lecture Notes in Computer Science}, %éditeurs : David Gilbert et Monika Heiner, %\newline{}
%     2012
%     %in: \textit{Computational Methods in Systems Biology}, éditeurs : David Gilbert et Monika Heiner, %\newline{}
%     %166--186, Springer Berlin Heidelberg, octobre 2012, DOI 10.1007/978-3-642-33636-2\_11.
%   \item \tval{Abducing Biological Regulatory Networks from Process Hitting models},\\
%     \textit{ECML-PKDD'12 / LDSSB'12}, %éditeurs : Oliver Ray et Katsumi Inoue,
%     2012
%     %24--35, septembre 2012.
% \end{itemize}
% 
% \emphcolor{Soumissions de journaux en cours :}
% \begin{itemize}
%   \item \tval{Folschette}, Paulevé, Magnin, Roux :\\
%     \tval{Sufficient Conditions for Reachability in Automata Networks with Priorities},\\
%     %version étendue de “Under-approximation of reachability in multivalued asynchronous networks”,
%     \emphcolor{soumis} à un numéro spécial de  de \textit{Theoretical Computer Science}% en avril 2014.
%   \item \tval{Folschette}, Paulevé, Inoue, Magnin, Roux :\\
%     \tval{Constructing Biological Regulatory Networks from Process Hitting models},\\
%     %extended version of “Concretizing the process hitting into biological regulatory networks”,
%     \emphcolor{en cours de révision} pour \textit{Theoretical Computer Science}
%   \item Paulevé, \tval{Folschette}, Magnin, Roux :\\
%     \tval{Analyses statiques de la dynamique des réseaux d'automates indéterministes},\\
%     \emphcolor{soumis} à un numéro spécial de \textit{Technique et Science Informatiques}
%     %editors: N.~Fatès and S.~Sené,
%     %\emphcolor{soumis} en avril 2014.
% \end{itemize}
% 
% \end{frame}
