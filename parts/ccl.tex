% Conclusion

\begin{frame}[c]
  \frametitle{Conclusion}

The Process Hitting allows to represent biological regulatory networks:
\begin{itemize}
  \item Qualitative and atomistic modeling
  \item Existing efficient \tval{reachability analysis}
  \item Links with other formalisms \f esp. from Thomas' modeling
\end{itemize}

\medskip
The \tval{impact degree}:
\begin{itemize}
  \item Quantification of the importance of a component
  \item Highlights possible recovery paths
  \item But limited to the presence/absence of a component
\end{itemize}

\medskip
Quantification of the perturbation using Process Hitting:
\begin{itemize}
  \item Adapted notion of \tval{impact degree} (multiple values)
  \item Thanks to the powerful \tval{reachability analysis}
  \item Additional properties with the graph of local causality
\end{itemize}
\end{frame}



\begin{frame}[c]
  \frametitle{Outlooks}

Search for alternative paths...
\begin{itemize}
  \item ...and their costs
  \item ...or implied delays
\end{itemize}

\pause
\vspace*{2cm}
\raggedleft
\Large
\bfseries
Thank you!\qquad~\\~\\
Any ideas?\qquad~\\~\\
Any questions?\qquad~\\

\end{frame}
