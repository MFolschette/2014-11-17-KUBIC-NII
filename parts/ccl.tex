% Conclusion

\begin{frame}[c]
  \frametitle{Conclusion}

The Process Hitting allows to represent biological regulatory networks:
\begin{itemize}
  \item Qualitative and atomistic modeling
  \item Existing efficient \tval{reachability analysis}
  \item Links with other formalisms \f esp. from Thomas' modeling
\end{itemize}

\pause
\bigskip
The \tval{impact degree}:
\begin{itemize}
  \item Quantification of the importance of a component
  \item Highlights possible recovery paths
  \item But limited to the presence/absence of a component
\end{itemize}

\pause
\bigskip
Quantification of the perturbation using Process Hitting:
\begin{itemize}
  \item Adapted notion of \tval{impact degree} (multiple values)
  \item Thanks to the powerful \tval{reachability analysis}
  \item Additional properties with the graph of local causality
\end{itemize}
\end{frame}



\begin{frame}[c]
\vspace*{2cm}
\centering
\Large
\bfseries
\textcolor{couleurtheme}{
Thank you!\qquad~\\~\\~\\~\\
Do you have questions\qquad~\\~\\
or suggestions?\qquad~\\
}
\end{frame}
