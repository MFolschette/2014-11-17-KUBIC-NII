% Bibliographie

\begin{frame}[c]
\frametitle{Bibliography}

\small
%\setlength{\parindent}{-1em}
\setlength{\parskip}{0.5em}

\tcitebullet Hao Jiang, Takeyuki Tamura, Wai-Ki Ching and Tatsuya Akutsu.
\emphcolor{On the Complexity of Inference and Completion of Boolean Networks from Given Singleton Attractors}.
In IEICE Transactions on Fundamentals of Electronics, \textit{Communications and Computer Sciences} E96.A, n. 11, pages 2265--74, 2013.

\tcitebullet Jean-Paul Comet and Gilles Bernot.
\emphcolor{Introducing continuous time in discrete models of gene regulatory networks}.
In \textit{Proceedings of the Nice Spring school on Modelling and simulation of biological processes in the context of genomics},
EDP Sciences, 2010.

\end{frame}



% \begin{frame}[c]
% \frametitle{Bibliography}
% 
% %\footnotesize
% \small
% \setlength{\parindent}{-1em}
% \setlength{\parskip}{0.5em}
% 
% \tcitebullet Adrien Richard, Jean-Paul Comet, Gilles Bernot.
% \emphcolor{R. Thomas' logical method}, 2008.
% Invité à \textit{Tutorials on modelling methods and tools: Modelling a genetic switch and Metabolic Networks}, Spring School on Modelling Complex Biological Systems in the Context of Genomics.
% 
% \tcitebullet Stuart A. Kauffman.
% \emphcolor{Metabolic stability and epigenesis in randomly constructed genetic nets}.
% \textit{Journal of Theoretical Biology}, 22(3), pages 437--467, 1969.
% 
% \tcitebullet René Thomas.
% \emphcolor{Boolean formalization of genetic control circuits}.
% \textit{Journal of Theoretical Biology}, 42(3), pages 563--585, 1973.
% 
% \tcitebullet Élisabeth Remy, Paul Ruet and Denis Thieffry.
% \emphcolor{Graphic requirements for multistability and attractive cycles in a boolean dynamical framework}.
% \textit{Advances in Applied Mathematics}, 41(3), pages 335--350, Elsevier, 2008.
% 
% \tcitebullet Adrien Richard, Jean-Paul Comet.
% \emphcolor{Necessary conditions for multistationarity in discrete dynamical systems}.
% \textit{Discrete Applied Mathematics}, 155(18), pages 2403--2413, 2007.
% 
% \tcitebullet Gilles Bernot, Jean-Paul Comet, Adrien Richard and Janine Guespin.
% \emphcolor{Application of formal methods to biological regulatory networks: extending Thomas' asynchronous logical approach with temporal logic}.
% \textit{Journal of Theoretical Biology}, 229(3), pages 339--347, Elsevier, 2004.
% 
% \tcitebullet Sohei Ito, Naoko Izumi, Shigeki Hagihara and Naoki Yonezaki.
% \emphcolor{Qualitative analysis of gene regulatory networks by satisfiability checking of Linear Temporal Logic}.
% In 2010 IEEE International Conference on \textit{BioInformatics and BioEngineering}, pages 232--237, IEEE, 2010.
% 
% \end{frame}
% 
% 
% 
% \begin{frame}[c]
%   \frametitle{Bibliographie}
% 
% \small
% \setlength{\parindent}{-1em}
% \setlength{\parskip}{0.5em}
% 
% \tcitebullet Loïc Paulevé, Morgan Magnin, Olivier Roux.
% \emphcolor{Refining dynamics of gene regulatory networks in a stochastic $\pi$-calculus framework}.
% In Corrado Priami, Ralph-Johan Back, Ion Petre, and Erik de Vink, editors: Transactions on Computational Systems Biology XIII,
% \textit{Lecture Notes in Computer Science} 6575, pages 171--191, 2011.
% 
% \tcitebullet Loïc Paulevé, Morgan Magnin, Olivier Roux.
% \emphcolor{Static analysis of biological regulatory networks dynamics using abstract interpretation}.
% \textit{Mathematical Structures in Computer Science}, 2012.
% 
% \tcitebullet Paul François, Vincent Hakim, Eric D Siggia.
% \emphcolor{Deriving structure from evolution : metazoan segmentation}.
% \textit{Molecular Systems Biology}, 3(1), 2007.
% 
% \tcitebullet Özgür Sahin \textit{et al.}
% \emphcolor{Modeling ERBB receptor-regulated G1/S transition to find novel targets for de novo trastuzumab resistance}.
% \textit{BMC Systems Biology}, 3(1), 2009.
% 
% \tcitebullet Regina Samaga \textit{et al.}
% \emphcolor{The Logic of EGFR/ErbB Signaling: Theoretical Properties and Analysis of High-Throughput Data}.
% \textit{PLoS Computational Biology}, 5(8), 2009.
% 
% \tcitebullet Steffen Klamt \textit{et al.}
% \emphcolor{A methodology for the structural and functional analysis of signaling and regulatory networks}.
% \textit{BMC Bioinformatics}, 7(1), 2006.
% 
% \tcitebullet Julio Saez-Rodriguez \textit{et al.}
% \emphcolor{A Logical Model Provides Insights into T Cell Receptor Signaling}.
% \textit{PLoS Computational Biology}, 3(8), 2007.
% 
% \end{frame}

%\tcitebullet Adrien Richard, Jean-Paul Comet, Gilles Bernot. \textit{Modern Formal Methods and App.}, chapter \emphcolor{Formal Methods for Modeling Biological Regulatory Networks}, pages 83--122, 2006.

%\tcitebullet Maxime Folschette, Loïc Paulevé, Katsumi Inoue, Morgan Magnin, Olivier Roux. \emphcolor{Concretizing the Process Hitting into Biological Regulatory Networks}. In David Gilbert and Monika Heiner, editors, \textit{Computational Methods in Systems Biology X}, Lecture Notes in Computer Science, pages 166--186. Springer Berlin Heidelberg, 2012.

%\tcitebullet Maxime Folschette, Loïc Paulevé, Morgan Magnin, Olivier Roux. \emphcolor{Under-approximation of Reachability in Multivalued Asynchronous Networks}. In E. Merelli and A. Troina, editors, \textit{4th International Workshop on Interactions between Computer Science and Biology (CS2Bio’13)}, Electronic Notes in Theoretical Computer Science, Volume 299, 33–51. June 2013.

%\tcitebullet Loïc Paulevé. PhD thesis: \emphcolor{\textit{Modélisation, Simulation et Vérification des Grands Réseaux de Régulation Biologique}}, October 2011, Nantes, France.

%\tcitebullet Loïc Paulevé, Morgan Magnin, and Olivier Roux. \textit{Tuning Temporal Features within the Stochastic $\pi$-Calculus}. IEEE Transactions on Software Engineering, 37(6), pages 858--871, 2011.

%\tcitebullet Loïc Paulevé and Adrien Richard. \textit{Topological Fixed Points in Boolean Networks}. Comptes Rendus de l'Académie des Sciences - Series I - Mathematics, 348(15-16), pages 825--828, 2010.

%\tcitebullet Gilles Bernot, Franck Cassez, Jean-Paul Comet, Franck Delaplace, Céline Müller, Olivier Roux. \emphcolor{Semantics of Biological Regulatory Networks}. \textit{Proceedings of the First Workshop on Concurrent Models in Molecular Biology}, Electronic Notes in Theoretical Computer Science 180(3), pages 3--14, 2007.

%\tcitebullet Adrien Richard. \emphcolor{Negative circuits and sustained oscillations in asynchronous automata networks}. \textit{Advances in Applied Mathematics} 44(4), pages 378--392, 2010.
