% Exemples de structures abstraites (graphes de causalité locale)

\begin{frame}
  \frametitle{Under-approximation}

\def \tu {2}
\def \tub {3}
\def \tuf {4}

\begin{columns}
\begin{column}{0.48\textwidth}

\begin{center}
\scalebox{0.55}{
\begin{tikzpicture}
  \exatt
  \TState{-\tu}{a_1,b_1,c_1,d_0}
  \TState{\tub-}{a_0,b_1,c_0,d_0}
  \node[objective] (d_2) at (d_2.center) {?};
\end{tikzpicture}
}
\end{center}

\end{column}
\begin{column}{0.52\textwidth}

\vspace{1.5em}
\tval{Sufficient condition}:

\smallskip
\begin{itemize}
  \item no cycle
  \item \only<-\tu>{each objective has a solution} \only<\tub->{\sout{each objective has a solution}}
\end{itemize}
\begin{center}
  \only<\tu>{\Large\textcolor{darkgreen}{$R$ is \textbf{true}}} \only<\tuf>{\Large\textcolor{darkyellow}{\textbf{Inconclusive}}}
\end{center}

\end{column}
\end{columns}

\begin{center}%
%\vspace*{1cm}%
\scalebox{\scaleex}{%
\only<-\tu>{%
\scalebox{\scaleex}{%
\begin{tikzpicture}[aS]
  \path[use as bounding box] (.7,1) rectangle (5.8,2.5);

  \glclegend{}{$d_2$}{$\PHobj{d_0}{d_2}$}
\end{tikzpicture}
}
  \sauyes
}
\only<\tub->{
  \sauinconc
}}
\end{center}
\end{frame}
