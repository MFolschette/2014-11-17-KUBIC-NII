% Définition du Process Hitting + sortes coopératives

\begin{frame}[c]
  \frametitle{Process Hitting}
  \framesubtitle{\tcite{\cpmrtcsb}}

The \tval{Process Hitting} is:
\begin{itemize}
  \item A recent formalism well-adapted to the modeling of BRNs
  \item An \tval{atomistic, qualitative and asynchronous} modeling (explicit \& discrete expression levels)
  \item \tval{Simple but powerful} dynamics (constraints on the form of actions)
\end{itemize}

\pause
\bigskip
Previously developed tools:
\begin{itemize}
  \item \tval{Reachability analysis} by abstract interpretation
  \item Fixed points enumeration
  \item Stochastic parameters
\end{itemize}

\pause
\medskip
\f The \tval{reachability analysis} is very efficient (polynomial time)

\f The Process Hitting is also well-adapted to study \tval{large BRNs}

\end{frame}



\begin{frame}[t]
  \frametitle{Standard Process Hitting}
  \framesubtitle{\tcite{\cpmrtcsb}}

% 1 : Sortes
\only<1>{
\tikzstyle{process}=[circle,minimum size=15pt,font=\footnotesize,inner sep=1pt]
\tikzstyle{tick label}=[color=white, font=\footnotesize]
\tikzstyle{tick}=[transparent]
\tikzstyle{hit}=[transparent]
\tikzstyle{selfhit}=[transparent, min distance=30pt,curve to]
\tikzstyle{bounce}=[transparent]
\tikzstyle{hlhit}=[transparent]
\begin{center}\scalebox{\scaleex}{
\begin{tikzpicture}
  \exphdef
\end{tikzpicture}
}\end{center}
}

% 2 : Processus
\only<2>{
\tikzstyle{process}=[circle,draw,minimum size=15pt,font=\footnotesize,inner sep=1pt]
\tikzstyle{tick label}=[font=\footnotesize]
\tikzstyle{tick}=[densely dotted]
\tikzstyle{hit}=[transparent]
\tikzstyle{selfhit}=[transparent, min distance=30pt,curve to]
\tikzstyle{bounce}=[transparent]
\tikzstyle{hlhit}=[transparent]
\begin{center}\scalebox{\scaleex}{
\begin{tikzpicture}
  \exphdef
\end{tikzpicture}
}\end{center}
}

% 3 : États
\only<3>{
\tikzstyle{hit}=[transparent]
\tikzstyle{selfhit}=[transparent, min distance=30pt,curve to]
\tikzstyle{bounce}=[transparent]
\tikzstyle{hlhit}=[transparent]
\begin{center}\scalebox{\scaleex}{
\begin{tikzpicture}
  \exphdef

  \TState{3}{a_0,b_1,z_0}
\end{tikzpicture}
}\end{center}
}

% 4 : Actions
\only<4->{
\tikzstyle{tick}=[densely dotted]
\tikzstyle{hit}=[->,>=angle 45]
\tikzstyle{selfhit}=[min distance=30pt,curve to]
\tikzstyle{bounce}=[densely dotted,>=stealth',->]
\tikzstyle{hlhit}=[very thick]
\begin{center}\scalebox{\scaleex}{
\begin{tikzpicture}
\exphdef
  \TState{4-5}{a_0,b_1,z_0}
  \TState{6}{a_0,b_1,z_1}
  \TState{7}{a_1,b_1,z_1}
  \TState{8}{a_1,b_1,z_2}

  \only<5>{
    \THit{b_1}{hl}{z_0}{.west}{z_1}
    \path[bounce,bend left,hl] \TBounce{z_0}{}{z_1}{.south};
  }
  \only<6>{
    \THit{a_0}{out=250,in=200,selfhit,hl}{a_0}{.west}{a_1}
    \path[bounce,bend left,hl] \TBounce{a_0}{}{a_1}{.south};
  }
  \only<7>{
    \THit{a_1}{hl}{z_1}{.west}{z_2}
    \path[bounce,bend left,hl] \TBounce{z_1}{}{z_2}{.south};
  }
\end{tikzpicture}
}\end{center}
}

\medskip
\begin{liste}
  \item \tval{Sorts}: components \qex{$a$, $b$, $z$}
\pause[2]
  \item \tval{Processes}: local states / discrete expression levels \qex{$z_0$, $z_1$, $z_2$}
\pause[3]
  \item \tval{States}: sets of active processes%
  \only<3-5>{\qex{$\PHetat{a_0, b_1, z_0}$}}%
  \only<6>{\qex{$\PHetat{a_0, b_1, z_1}$}}%
  \only<7>{\qex{$\PHetat{a_1, b_1, z_1}$}}%
  \only<8>{\qex{$\PHetat{a_1, b_1, z_2}$}}%
\pause[4]
  \item \tval{Actions}: dynamics \qex{\only<5>{\underline}{$\PHfrappe{b_1}{z_0}{z_1}$}, \only<6>{\underline}{$\PHfrappe{a_0}{a_0}{a_1}$}, \only<7>{\underline}{$\PHfrappe{a_1}{z_1}{z_2}$}}
\end{liste}
\end{frame}
